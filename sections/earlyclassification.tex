
\begin{frame}
	\frametitle{Winter Research Stay at IRISA Obelix Lab in France}
	
	\begin{columns}
		\column{.5\textwidth}
		
		Research Stay
		\begin{itemize}[itemsep=1em]
			\item Prof. \textbf{Sebastien Lefèvre} and Prof. \textbf{Romain Tavenard}
			\item Obelix: \textbf{Environment observation} with \textbf{complex imagery}
			\item Vannes and Rennes, Britanny, France
		\end{itemize}
		
		\vspace{1em}
		\url{http://www-obelix.irisa.fr/}
		
		\column{.5\textwidth}
		
		\includegraphics[width=5cm]{images/map/europe}
		
	\end{columns}
	
\end{frame}
%
%\begin{frame}<presentation:1>
%\frametitle{Auto-regressive classification model}
%
%\begin{columns}
%	
%	\column{.5\textwidth}
%	\begin{center}
%		\input{images/backprop_example.tikz}
%	\end{center}
%	\column{.5\textwidth}
%	\input{images/qualitative_example.tikz}
%	
%	
%\end{columns}
%
%\end{frame}
%
%\begin{frame}
%	\frametitle{Research Stay in Brittany}
%	
%	
%\end{frame}


\tikzstyle{rnn}=[draw,circle]
\tikzstyle{annot}=[rounded corners, fill=colorblue!20]
\colorlet{colortrain}{colorblue}
\colorlet{colorinfer}{colororange}
\tikzstyle{infer}=[-stealth, shorten >=.0em, shorten <=.0em, colorinfer]
\tikzstyle{loss}=[fill=colorblue!10, rounded corners, font=\small]
\tikzstyle{grad}=[colortrain]

\newcommand{\classimagepair}[1]{
	\def\sample{#1}
	\begin{tikzpicture}[node distance=.2em]
	\node[label=above:{inputs $\V{x}_t$}](a){\includegraphics[width=.5\textwidth]{images/classification_without_earliness/TwoPatterns30EpochsNoEarliness/sample_\sample_x.png}};
	\node[label=above:{softmaxed class scores $\yhat_t$}, right=of a](b){\includegraphics[width=.5\textwidth]{images/classification_without_earliness/TwoPatterns30EpochsNoEarliness/sample_\sample_p(y).png}};
	
	\visible<2->{
		%\draw (-2,-2) to[grid with coordinates] (8,4);
		\node[annot, yshift=3em](wiggle1) at ($ (a)!0.3!(b) $) {event \#1};
		\draw (wiggle1) -- (-1.5,0);
		\draw (wiggle1) -- (5.5,-1);
	}
	\visible<3->{
		\node[annot, yshift=-5em](wiggle2) at ($ (a)!0.3!(b) $) {event \#2};
		\draw (wiggle2) -- (1,0);
		\draw (wiggle2) -- (7.5,0);
	}
	
	\visible<4->{
		\draw[very thick] (8.5,-2) -- (8.5,2); 
		\node[annot, yshift=-5em, anchor=east](stop) at (10,-1) {...we could stop here};
%		\draw[shorten >=1em] (stop)++(2,0.5) -- (8,0);
	}
	\end{tikzpicture}
}

\begin{frame}<presentation:1-4>{Class Predictions}
\classimagepair{0}
\end{frame}

\begin{frame}
\frametitle{ArXiv Paper: End-to-end Learning for Early Classification of Time Series}

\begin{columns}
	
	
	\column{.5\textwidth}
	
	Rußwurm, M., Lefèvre, S., Courty, N., Emonet, R., Körner, M., \& Tavenard, R. (2019).\textbf{ End-to-end Learning for Early Classification of Time Series}. arXiv preprint arXiv:1901.10681.
	
	
	\column{.5\textwidth}
	
	\includegraphics[width=4cm]{images/elects_arxiv}
	
\end{columns}

\end{frame}



%\begin{frame}
%	\frametitle{Autoregressive Classification Model}
%	
%	\input{images/classmodel.tikz}
%	
%\end{frame}


\begin{frame}
\frametitle{Early Classification on Remote Sensing Data}
\input{images/example.tikz}

\url{https://arxiv.org/abs/1901.10681}


\end{frame}

\begin{frame}
	\frametitle{Stopping times per crop Class}
	
	\input{images/classboxplots.tikz}
	
\end{frame}

\begin{frame}
	\frametitle{Impact of Early Classification on Vegetation Data}
	
	\Large
	
	\begin{itemize}[itemsep=1em]
		\item<1-> \textbf{supervised end-to-end} learning scenario
		\item<2-> we get a stopping time \textbf{for free} solely from classifying labels
		\item<3-> relate to \textbf{characteristic features}, i.e., \textbf{crop phenology}
		\item<4-> next: assess seasonal shifts in \textbf{vegetation phenology} due to \textbf{environmental conditions}
	\end{itemize}
	
\end{frame}

