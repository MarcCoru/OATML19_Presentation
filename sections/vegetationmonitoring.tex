
\begin{frame}
\frametitle{Multi-temporal Vegetation Monitoring}

\begin{columns}
	\column{.5\textwidth}
	
	\begin{tikzpicture}
		\node[] at (0,0){\includegraphics[width=\textwidth]{images/Large1954_cerial_growth_stages}};
		
%		\draw[step=1.0,black,thin, fill=none] (-2,-2) grid (2,2);
		
		\visible<-1>{\draw [fill=white, draw=none, opacity=0.8] (-0.8,-3) rectangle (2,2.5);}
		\visible<-2>{\draw [fill=white, draw=none, opacity=0.8] (2,-3) rectangle (5,2.5);}
		
		\visible<1>{\node[rotate=190] at (-2.5,1.5){\includegraphics[width=15mm]{images/icons/sat2}};}
		\visible<2>{\node[rotate=225] at (-2.5,1.5){\includegraphics[width=15mm]{images/icons/sat2}};}
		\visible<3->{\node[rotate=260] at (-2.5,1.5){\includegraphics[width=15mm]{images/icons/sat2}};}
		
		
		\visible<4->{\node at (-1.5,1.4) {\includegraphics[width=10mm]{images/cloud}};
		}
		
	\end{tikzpicture}
	
	\column{.5\textwidth}
	
	{\Large
	\only<1>{
	\begin{equation*}
		\V{y} = f_\text{phenology}(\V{X}_t)
	\end{equation*}
	}
	\only<2>{
		\begin{equation*}
		\V{y} = f_\text{phenology}(\V{X}_t,\V{X}_{t+1})
		\end{equation*}
	}
	\only<3>{
		\begin{equation*}
		\V{y} = f_\text{phenology}(\V{X}_t,\V{X}_{t+1},\V{X}_{t+2})
		\end{equation*}
	}
	}
	
	
	\vspace{2em}

	
	\visible<1->{\includegraphics[width=.22\textwidth]{images/s2grid/1}}
	\visible<2->{\includegraphics[width=.22\textwidth]{images/s2grid/2}}
	\visible<3->{\includegraphics[width=.22\textwidth]{images/s2grid/3}}
	\visible<4->{\includegraphics[width=.22\textwidth]{images/s2grid/4}}
	
	\vspace{1em}
	
	{\small 
		Large, E. C. (1954). Growth stages in cereals illustration of the Feekes scale. Plant pathology, 3(4), 128-129.
	}
	
	
\end{columns}



\end{frame}



\begin{frame}[t]
\frametitle{Looking at Sequence to Sequence Models from NLP}
%	\framesubtitle{Natürlicher Sprachverarbeitung, Übersetzung, Spracherkennung}

%Verbreitet in sequenziellen Aufgabengebieten, wie natürlicher Sprachverarbeitung, Übersetzung, Spracherkennung
%	\begin{center}
%		\large Wort-Sequenz $\rightarrow$ Representation $c$ $\rightarrow$ Wort-Sequenz
%	\end{center}
\begin{center}
	\input{images/seq2seq.tikz}
\end{center}
%\begin{columns}
%	\column{.4\textwidth}
%	%		\brand{Word2Vec} embedding 
%	%		Neural Machine Translation encodes 
%	
%	\centering 
%%	
%%	\begin{tikzpicture}
%%	\node[fill=tumbluelight!50,rounded corners](x){Eingabesequenz};
%%	\node[fill=tumbluelight!50,rounded corners, below=of x](c){Repräsentation};
%%	\node[fill=tumbluelight!50,rounded corners, below=of c](o){Ausgabsequenz};
%%	
%%	\draw[-stealth] (x) -- (c);
%%	\draw[-stealth] (c) -- (o);
%%	
%%	\end{tikzpicture}
%%	
%	\column{.6\textwidth}
%	
%\end{columns}


{\small
	Sutskever, I., Vinyals, O., \& Le, Q. V. (2014). Sequence to sequence learning with neural networks. In Advances in neural information processing systems (pp. 3104-3112).}


\end{frame}

%\tikzsetnextfilename{network}

\tikzstyle{operator} = [draw, circle, fill=tumbluemedium, draw=tumbluemedium, inner sep=0, text=white]
%\tikzstyle{function} = [draw, rectangle, fill=tumbluemedium, draw=tumbluemedium, text=white]
\tikzstyle{gate} = [fill=tumivory,draw,rounded corners]

\tikzstyle{dummy} = [inner sep=0]
\tikzstyle{flow} = [rounded corners]
\tikzstyle{endflow} = [-stealth,flow]
\tikzstyle{beginflow} = [stealth-,flow]

\tikzstyle{bigpassbox} = [opacity=.2, rounded corners, draw=none]

% defaultvalue -> might be replaced later
\colorlet{tensorcolor}{tumblue}

\tikzstyle{bigbox} = [rectangle, draw=tumivory, thick, fill=tumgraylight, rounded corners, 
inner sep=.5ex]

\tikzstyle{wireframe} = [draw=tumgray]
\tikzstyle{image} = [inner sep=0, fill=none, minimum size=\imagewidth]

\tikzset{pic shift/.store in=\shiftcoord,
	pic shift={(0,0)},
	pics/seqlstmfw/.style={
		code={
		\begin{scope}[shift={\shiftcoord},xscale=1.3,yscale=.8]
			
			\node[dummy] (bl) at (0,0){}; % bottom left
			\node[dummy] (tr) at (1,1){}; % top right
			
			\node[dummy] (br) at ($ (bl -| tr) $){}; % bottom right
			\node[dummy] (tl) at ($ (bl |- tr) $){}; % top left
			
			\node[fit=(bl) (tr),bigbox] (-C) {};
			
			% input coordinate for rounded draw lines -> slightly right of tl
			\coordinate (-input) at (0.1,1); % top left
			
			% output coordinate for rounded draw lines -> slightly left of br
			\coordinate (-coutput) at (0.9,0); % bottom right
			\coordinate (-cinput) at (0.1,0); % bottom left
			\coordinate (-houtput) at (0.9,1); % bottom right
			
%			% gate distance
			\def\d{1/6}
			
			% gate heights
			\def\h{1/3}
			
			\coordinate (f)  at bl+(1*\d,0);
			\coordinate (i)  at bl+(2*\d,0);
			\coordinate (j)  at bl+(3*\d,0);
			\coordinate (o)  at bl+(4*\d,0);
			\coordinate (out) at bl+(5*\d,0);
			
			\coordinate (gates) at (0,2*\h);
			
			%\node[above=of tl](xt){$x_{t}$};
			%\node[left=of tl](htminus1){$h_{t-1}$};
			
			%\node[below=of br](ct){$c_{t}$};
			
			\node[gate](fgate) at ($ (gates -| f) $){};
			\node[gate](igate) at ($ (gates -| i) $){};
			\node[gate](jgate) at ($ (gates -| j) $){};
			\node[gate](ogate) at ($ (gates -| o) $){};
			
%			\coordinate (htminus1) at bl+(-.5,0);
%			\coordinate (ht) at bl+(-.5,0);
%			
			% forget gate
			\node[operator](fmult) at ($ (bl -| fgate) $) {};
			\draw[endflow] (-input) -| (fgate) -- (fmult); 
			
%			%j
			\node[operator](jmult) at ([shift={(0,-1*\h)}]jgate) {};
			\node[operator](cadd) at ($ (bl -| jgate) $) {};
			\draw[endflow] (-input) -| (jgate) -- (jmult);
			\draw[endflow] (jmult) -- (cadd); 			

%			%i	
			\draw[endflow] (-input) -| (igate) |- (jmult); 
%
%%			% outpu
			\node[operator](outtanh) at ([shift={(0,1*\h)}]out) {};
%			
%			%o 
			\draw[endflow] (tl) -| (ogate) |- (outtanh);
			\draw[flow] (outtanh) |- (-houtput);
%			
%			% output flow
			\draw[endflow] (cadd) -| (outtanh);
			\draw[flow] (-cinput) -- (fmult) -- (cadd) -- (-coutput);
%			
			
			% debug
%			\node at (gates) {\tiny{gates}};
%			\node at (-input) {\tiny{-input}};
%			\node at (-coutput) {\tiny{-coutput}};
%			\node at (-houtput) {\tiny{-houtput}};
%			\node at (f) {\tiny{f}};
%			\node at (i) {\tiny{i}};
%			\node at (j) {\tiny{j}};
%			\node at (o) {\tiny{o}};
%			\node at (tl) {\tiny{tl}};
%			\node at (br) {\tiny{br}};
%			\node at (bl) {\tiny{bl}};
%			\node at (tr) {\tiny{tr}};
%			\node at (out) {\tiny{out}};
			
		\end{scope}
		}
	}
}

\tikzset{pic shift/.store in=\shiftcoord,
	pic shift={(0,0)},
	pics/seqlstmbw/.style={
		code={
			\begin{scope}[shift={\shiftcoord},xscale=1.3,yscale=-.8]
				
				\node[dummy] (bl) at (0,0){}; % bottom left
				\node[dummy] (tr) at (1,1){}; % top right
				
				\node[dummy] (br) at ($ (bl -| tr) $){}; % bottom right
				\node[dummy] (tl) at ($ (bl |- tr) $){}; % top left
				
				\node[fit=(bl) (tr),bigbox] (-C) {};
				
				% input coordinate for rounded draw lines -> slightly right of tl
				\coordinate (-input) at (0.1,1); % top left
				
				% output coordinate for rounded draw lines -> slightly left of br
				\coordinate (-coutput) at (0.9,0); % bottom right
				\coordinate (-cinput) at (0.1,0); % bottom left
				\coordinate (-houtput) at (0.9,1); % top right
				
				%			% gate distance
				\def\d{1/6}
				
				% gate heights
				\def\h{1/3}
				
				\coordinate (f)  at bl+(1*\d,0);
				\coordinate (i)  at bl+(2*\d,0);
				\coordinate (j)  at bl+(3*\d,0);
				\coordinate (o)  at bl+(4*\d,0);
				\coordinate (out) at bl+(5*\d,0);
				
				\coordinate (gates) at (0,2*\h);
				
				%\node[above=of tl](xt){$x_{t}$};
				%\node[left=of tl](htminus1){$h_{t-1}$};
				
				%\node[below=of br](ct){$c_{t}$};
				
				\node[gate](fgate) at ($ (gates -| f) $){};
				\node[gate](igate) at ($ (gates -| i) $){};
				\node[gate](jgate) at ($ (gates -| j) $){};
				\node[gate](ogate) at ($ (gates -| o) $){};
				
				%			\coordinate (htminus1) at bl+(-.5,0);
				%			\coordinate (ht) at bl+(-.5,0);
				%			
				% forget gate
				\node[operator](fmult) at ($ (bl -| fgate) $) {};
				\draw[endflow] (-input) -| (fgate) -- (fmult); 
				
				%			%j
				\node[operator](jmult) at ([shift={(0,-1*\h)}]jgate) {};
				\node[operator](cadd) at ($ (bl -| jgate) $) {};
				\draw[endflow] (-input) -| (jgate) -- (jmult);
				\draw[endflow] (jmult) -- (cadd); 			
				
				%			%i	
				\draw[endflow] (-input) -| (igate) |- (jmult); 
				%
				%%			% outpu
				\node[operator](outtanh) at ([shift={(0,1*\h)}]out) {};
				%			
				%			%o 
				\draw[endflow] (tl) -| (ogate) |- (outtanh);
				\draw[flow] (outtanh) |- (-houtput);
				%			
				%			% output flow
				\draw[endflow] (cadd) -| (outtanh);
				\draw[flow] (-cinput) -- (fmult) -- (cadd) -- (-coutput);
				%			
				
				% debug
				%			\node at (gates) {\tiny{gates}};
				%			\node at (-input) {\tiny{-input}};
				%			\node at (-coutput) {\tiny{-coutput}};
				%			\node at (-houtput) {\tiny{-houtput}};
				%			\node at (f) {\tiny{f}};
				%			\node at (i) {\tiny{i}};
				%			\node at (j) {\tiny{j}};
				%			\node at (o) {\tiny{o}};
				%			\node at (tl) {\tiny{tl}};
				%			\node at (br) {\tiny{br}};
				%			\node at (bl) {\tiny{bl}};
				%			\node at (tr) {\tiny{tr}};
				%			\node at (out) {\tiny{out}};
				
			\end{scope}
		}
	}
}

%\newcommand{}{
%	\begin{tikzpicture}
%	% each layer
%	\begin{scope}[scale=2]
%	
%	% raster size
%	\def\d{0.7}		
%	
%	% distance layer
%	\def\s{\d*5}
%	
%	\foreach \i in {1,...,6}
%	{		
%		\begin{scope}[
%		yshift=\s*\i,every node/.append style={
%			yslant=0.5,xslant=-1},yslant=0.5,xslant=-1
%		]
%		%\draw[step=3.33mm] (0,0) grid (1,1);
%		%\fill[black,fill opacity=.9] (0.333,0.333) rectangle (0.333,0.333);    	    	  
%		
%		\foreach \row in {0,...,2}{
%			\foreach \col in {0,...,2}{
%				\draw[tumblack, fill=tumblue!\pdfuniformdeviate 40,fill opacity=1,rounded corners=1] (\col*\d/3,\row*\d/3) rectangle (\col*\d/3+\d/3, \row*\d/3+\d/3);
%				%                 \draw[black, fill=black!\pdfuniformdeviate 40,fill opacity=1,rounded corners=1] (\col*\d/3,\row*\d/3) rectangle (\col*\d/3+\d/3, \row*\d/3+\d/3);
%			}
%		}
%		
%		%\draw[step=3.33mm] (0,0) grid (1,1);
%		%\fill[white,fill opacity=.9] (0,0) rectangle (1,1);
%		\end{scope}
%	}
%	\end{scope}
%	\end{tikzpicture}
%}
%%
%\newcommand{}{
%	
%	\begin{tikzpicture}[scale=2]
%	\foreach \i in {0,...,9}
%	\draw[tumblack,fill=tumorange!\pdfuniformdeviate 40,fill opacity=.9, rounded corners=0.5] (\i*.2, 0) rectangle (\i*.2+.2, .2);
%	\end{tikzpicture}
%}

\newcommand{\figencoderfieldRNN}{
	
	\begin{tikzpicture}[scale=1, node distance=1em]%,show background rectangle,background rectangle/.style={draw=red}]
	
	%\matrix (m) [matrix of nodes, ampersand replacement=\&]{
	
	\def\d{1.7}
	\def\encoderheight{0.4}
	\def\decoderheight{-0.4}
	\def\imagewidth{8mm}
	\def\stateimagewidth{8mm}
	\def\classimagewidth{8mm}
	
	\def\rgbone{images/network/48px/rgb1}
	\def\rgbtwo{images/network/48px/rgb2}
	\def\rgbthree{images/network/48px/rgb3}
	\def\rgbfour{images/network/48px/rgb4}
	\def\prediction{images/network/48px/prediction}
	\def\groundtruth{images/network/48px/ground_truth}
	\def\activation{images/network/48px/maize}
	\def\state{images/network/48px/state}
	
	
	%\draw [bigpassbox, fill=forwardcolor](1,.25) rectangle (10,4);
	
	\node[bigpassbox, fill=none, rectangle, minimum width=7cm,minimum height=3cm, anchor=south west, label={[shift={(2em,-3.5ex)}]north:}] (forwardbox) at (1,.15) {};
	%\draw [bigpassbox, fill=backwardcolor](1,-.25) rectangle (10,-4);
	
	\draw pic (fw1) at (\d,\encoderheight) {seqlstmfw};% \&
	\node[above=of fw1tl, label=above:{$\VInput_{1}$}](xfw1){
		
	};%images/network/time1_x
	
	\node[below= 3.5emof fw1-houtput, xshift=.5em, label=below:{$\V{y}_{1}$}](yfw1){
		
	};%images/network/time1_x
	
	\draw pic (fw2) at (2*\d,\encoderheight) {seqlstmfw};
	\node[above=of fw2-input, label=above:{$\VInput_{2}$}](xfw2){
		
	}; % images/network/time11_x
	
	\node[below= 3.5emof fw2-houtput, xshift=.5em, label=below:{$\V{y}_{2}$}](yfw2){
		
	};%images/network/time1_x
	
	\draw pic (fw3) at (3*\d,\encoderheight) {seqlstmfw};
	\node[above=of fw3-input](xfw3){$\dots$};
	\node[below= 3.5emof fw3-houtput, xshift=.5em](yfw3){
		%	
		$\dots$
	};%images/network/time1_x
	
	\draw pic (fw4) at (4*\d,\encoderheight) {seqlstmfw};
	\node[above=of fw4-input, label=above:{$\VInput_T$}](xfw4){
		
	}; % images/network/time30_x
	
	\node[below= 3.5emof fw4-houtput, xshift=.5em, label=below:{$\V{y}_{T}$}](yfw4){
		
	};%images/network/time1_x
	
	\node[left=of fw1-input](enczerostateh){$\V{0}$};
	\node[left=of fw1-cinput](enczerostatec){$\V{0}$};
	\draw[endflow] (enczerostateh) -- (fw1-input);
	\draw[endflow] (enczerostatec) -- (fw1-cinput);
	
	% draw connections from input to cells
	\draw[flow] (xfw1) |- (fw1-input);
	\draw[flow] (xfw2) |- (fw2-input);
	\draw[flow] (xfw3) |- (fw3-input);
	\draw[flow] (xfw4) |- (fw4-input);
	
	% draw hidden connections between cells
	\draw[endflow] (fw1-houtput) -- (fw2-input);
	\draw[endflow] (fw2-houtput) -- (fw3-input);
	\draw[endflow] (fw3-houtput) -- (fw4-input);
	
	% draw hidden connections between cells
	\draw[endflow] (fw1-coutput) -- (fw2-cinput);
	\draw[endflow] (fw2-coutput) -- (fw3-cinput);
	\draw[endflow] (fw3-coutput) -- (fw4-cinput);
	
	% outputs
	\draw[endflow] (fw1-houtput) -- ($ (fw1-houtput)+(.5em,0) $) -- (yfw1);
	\draw[endflow] (fw2-houtput) -- ($ (fw2-houtput)+(.5em,0) $) -- (yfw2);
	\draw[endflow] (fw3-houtput) -- ($ (fw3-houtput)+(.5em,0) $) -- (yfw3);
	\draw[endflow] (fw4-houtput) -- ($ (fw4-houtput)+(.5em,0) $) -- (yfw4);
	
	%\node[right= 2.5em of fw4-coutput](cfw){$\VCellState_T^\text{fw}$};
	
	%\draw[endflow] (fw4-coutput) -- (cfw);
	
	%};
	\end{tikzpicture}
}

\newcommand{\figencoder}{

\begin{tikzpicture}[scale=1, node distance=1em]%,show background rectangle,background rectangle/.style={draw=red}]

%\matrix (m) [matrix of nodes, ampersand replacement=\&]{

\def\d{1.7}
\def\encoderheight{0.4}
\def\decoderheight{-0.4}
\def\imagewidth{8mm}
\def\stateimagewidth{8mm}
\def\classimagewidth{8mm}

\def\rgbone{images/network/48px/rgb1}
\def\rgbtwo{images/network/48px/rgb2}
\def\rgbthree{images/network/48px/rgb3}
\def\rgbfour{images/network/48px/rgb4}
\def\prediction{images/network/48px/prediction}
\def\groundtruth{images/network/48px/ground_truth}
\def\activation{images/network/48px/maize}
\def\state{images/network/48px/state}


%\draw [bigpassbox, fill=forwardcolor](1,.25) rectangle (10,4);

\node[bigpassbox, fill=none, rectangle, minimum width=7cm,minimum height=3cm, anchor=south west, label={[shift={(2em,-3.5ex)}]north:}] (forwardbox) at (1,.15) {};
%\draw [bigpassbox, fill=backwardcolor](1,-.25) rectangle (10,-4);

\draw pic (fw1) at (\d,\encoderheight) {seqlstmfw};% \&
\node[above=of fw1tl, label=above:{$\VInput_{1}$}](xfw1){\includegraphics[width=\imagewidth]{\rgbone}};%images/network/time1_x

\draw pic (fw2) at (2*\d,\encoderheight) {seqlstmfw};
\node[above=of fw2-input, label=above:{$\VInput_{2}$}](xfw2){\includegraphics[width=\imagewidth]{\rgbtwo}}; % images/network/time11_x

\draw pic (fw3) at (3*\d,\encoderheight) {seqlstmfw};
\node[above=of fw3-input](xfw3){$\dots$};

\draw pic (fw4) at (4*\d,\encoderheight) {seqlstmfw};
\node[above=of fw4-input, label=above:{$\VInput_T$}](xfw4){\includegraphics[width=\imagewidth]{\rgbfour}}; % images/network/time30_x

\node[left=of fw1-input](enczerostateh){$\V{0}$};
\node[left=of fw1-cinput](enczerostatec){$\V{0}$};
\draw[endflow] (enczerostateh) -- (fw1-input);
\draw[endflow] (enczerostatec) -- (fw1-cinput);

% draw connections from input to cells
\draw[flow] (xfw1) |- (fw1-input);
\draw[flow] (xfw2) |- (fw2-input);
\draw[flow] (xfw3) |- (fw3-input);
\draw[flow] (xfw4) |- (fw4-input);

% draw hidden connections between cells
\draw[endflow] (fw1-houtput) -- (fw2-input);
\draw[endflow] (fw2-houtput) -- (fw3-input);
\draw[endflow] (fw3-houtput) -- (fw4-input);

% draw hidden connections between cells
\draw[endflow] (fw1-coutput) -- (fw2-cinput);
\draw[endflow] (fw2-coutput) -- (fw3-cinput);
\draw[endflow] (fw3-coutput) -- (fw4-cinput);

\node[right= 2.5em of fw4-coutput](cfw){$\VCellState_T^\text{fw}$};

\draw[endflow] (fw4-coutput) -- (cfw);

%};
\end{tikzpicture}

}
\begin{frame}
\frametitle{Temporal Vegetation Modelling with LSTMs}
\framesubtitle{CVPR Earthvision 2017}
\begin{columns}
	\column{.5\textwidth}
	\figencoderfieldRNN
	\column{.5\textwidth}
	\includegraphics[width=.8\textwidth]{images/confmat_fieldrnn}
%		images/stmelfadditional/Genauigkeiten/accuracies1}
\end{columns}

\vspace{1em}

\textsl{\small
	Rußwurm, M. and Körner, M. (2017). \textbf{Temporal Vegetation Modelling using Long Short-Term Memory Networks for Crop Identification from Medium-Resolution Multi-Spectral Satellite Images}. In IEEE/ISPRS EarthVision 2017 Workshop, Proceedings of the IEEE CVPR Workshops. \textbf{\color{tumorange} Best Paper Award}
}

\end{frame}



\def\fps{3}
\input{images/cells.tikz}

%\begin{frame}[t]
%	\frametitle{Convolutional Long-Short Term Memory Cells (ConvLSTM)}
%	
%	%	Reveal Sequence:
%	%	1: update, empty cell
%	%	2: forget gate
%	%	3: input gate + modulation gate + jmult
%	%	4: output gate
%	%	5: cell state
%	%	6: output
%	
%	\begin{columns}
%		\begin{column}{0.3\textwidth}
%			\begin{tabularx}{\textwidth}{l}
%				LSTM update: \\
%				$\VHidden_t,\VCellState_t  \leftarrow \VInput_t, \VHidden_{t-1}, \VCellState_{t-1}$\\
%				\\
%				\visible<2->{Internal gates:} \\
%				\visible<2->{$\VForgetGate_t = \sigma( \conv{\concat{ \VInput_t }{ \VHidden_{t-1} } }{ \MWeight_f } + 1 )$} \\
%				\visible<3->{$\VInputGate_t = \sigma( \conv{ \concat{ \VInput_t }{ \VHidden_{t-1} } }{ \MWeight_i } ) $} \\
%				\visible<3->{$\VModulationGate_t = \sigma( \conv{ \concat{ \VInput_t }{ \VHidden_{t-1} } }{ \MWeight_j } ) $} \\
%				\visible<4->{$\VOutputGate_t = \sigma( \conv{ \concat{ \VInput_t }{ \VHidden_{t-1} } }{ \MWeight_o} ) $} \\
%				\\
%				\visible<5->{Internal cell state:} \\
%				\visible<5->{$\VCellState_t = \VCellState_{t-1} \odot \VForgetGate_t + \VInputGate_t \odot \VModulationGate_t$} \\
%				\visible<6->{Output:} \\
%				\visible<6->{$\VHidden_t= \VOutputGate_t \odot \tanh(\VCellState_t) $} \\
%			\end{tabularx}
%		\end{column}
%		\begin{column}{0.7\textwidth}
%			\lstmexplain
%		\end{column}
%	\end{columns}
%\end{frame}

\begin{frame}[t]
\frametitle{Recurrent Convolutional Neural Networks}
\framesubtitle{Convolutional Long Short-Term Memory --- ConvLSTM (Hochreiter \& Schmidhuber, 1997)}
\begin{columns}
	\begin{column}{0.3\textwidth}
				
		
		\textbf{Convolutional LSTM:}
		{\scriptsize
			Xingjian, S. H. I., Chen, Z., Wang, H., Yeung, D. Y., Wong, W. K., \& Woo, W. C. (2015). Convolutional LSTM network: A machine learning approach for precipitation nowcasting. In Advances in neural information processing systems (pp. 802-810).
		}
		
		
%		\begin{tabularx}{\textwidth}{l}
%			LSTM update: \\
%			$\VHidden_t,\VCellState_t  \leftarrow \VInput_t, \VHidden_{t-1}, \VCellState_{t-1}$\\
%			\\
%			Internal gates: \\
%			$\VForgetGate_t = \sigma( \conv{\concat{ \VInput_t }{ \VHidden_{t-1} } }{ \MWeight_f } + 1 )$ \\
%			$\VInputGate_t = \sigma( \conv{ \concat{ \VInput_t }{ \VHidden_{t-1} } }{ \MWeight_i } ) $ \\
%			$\VModulationGate_t = \sigma( \conv{ \concat{ \VInput_t }{ \VHidden_{t-1} } }{ \MWeight_j } ) $ \\
%			$\VOutputGate_t = \sigma( \conv{ \concat{ \VInput_t }{ \VHidden_{t-1} } }{ \MWeight_o \odot \VCellState_{t-1} } ) $ \\
%			\\
%			Internal cell state: \\
%			$\VCellState_t = \VCellState_{t-1} \odot \VForgetGate_t + \VInputGate_t \odot \VModulationGate_t$ \\
%			Output: \\
%			$\VHidden_t= \VOutputGate_t \odot \tanh(\VCellState_t) $ \\
%		\end{tabularx}
	\end{column}
	\vspace{-2em}	
	\begin{column}{0.7\textwidth}
		
		\lstmanim
%\lstmexplain
	\end{column}
\end{columns}



\end{frame}



%\input{images/activations.tikz}
%\newcounter{tcounter}
%\begin{frame}
%	\frametitle{Gate Activations}
%	\framesubtitle{LSTM cell 3 of 256}
%	\figactivations{1}{3}
%\end{frame}

%\begin{frame}
%	\frametitle{Gate Activations}
%	\framesubtitle{LSTM cell 22 of 256}
%	\figactivations{1}{22}
%\end{frame}

%%\tikzsetnextfilename{network}

\tikzstyle{operator} = [draw, circle, fill=tumbluemedium, draw=tumbluemedium, inner sep=0, text=white]
%\tikzstyle{function} = [draw, rectangle, fill=tumbluemedium, draw=tumbluemedium, text=white]
\tikzstyle{gate} = [fill=tumivory,draw,rounded corners]

\tikzstyle{dummy} = [inner sep=0]
\tikzstyle{flow} = [rounded corners]
\tikzstyle{endflow} = [-stealth,flow]
\tikzstyle{beginflow} = [stealth-,flow]

\tikzstyle{bigpassbox} = [opacity=.2, rounded corners, draw=none]

% defaultvalue -> might be replaced later
\colorlet{tensorcolor}{tumblue}

\tikzstyle{bigbox} = [rectangle, draw=tumivory, thick, fill=tumgraylight, rounded corners, 
inner sep=.5ex]

\tikzstyle{wireframe} = [draw=tumgray]
\tikzstyle{image} = [inner sep=0, fill=none, minimum size=\imagewidth]

\tikzset{pic shift/.store in=\shiftcoord,
	pic shift={(0,0)},
	pics/seqlstmfw/.style={
		code={
		\begin{scope}[shift={\shiftcoord},xscale=1.3,yscale=.8]
			
			\node[dummy] (bl) at (0,0){}; % bottom left
			\node[dummy] (tr) at (1,1){}; % top right
			
			\node[dummy] (br) at ($ (bl -| tr) $){}; % bottom right
			\node[dummy] (tl) at ($ (bl |- tr) $){}; % top left
			
			\node[fit=(bl) (tr),bigbox] (-C) {};
			
			% input coordinate for rounded draw lines -> slightly right of tl
			\coordinate (-input) at (0.1,1); % top left
			
			% output coordinate for rounded draw lines -> slightly left of br
			\coordinate (-coutput) at (0.9,0); % bottom right
			\coordinate (-cinput) at (0.1,0); % bottom left
			\coordinate (-houtput) at (0.9,1); % bottom right
			
%			% gate distance
			\def\d{1/6}
			
			% gate heights
			\def\h{1/3}
			
			\coordinate (f)  at bl+(1*\d,0);
			\coordinate (i)  at bl+(2*\d,0);
			\coordinate (j)  at bl+(3*\d,0);
			\coordinate (o)  at bl+(4*\d,0);
			\coordinate (out) at bl+(5*\d,0);
			
			\coordinate (gates) at (0,2*\h);
			
			%\node[above=of tl](xt){$x_{t}$};
			%\node[left=of tl](htminus1){$h_{t-1}$};
			
			%\node[below=of br](ct){$c_{t}$};
			
			\node[gate](fgate) at ($ (gates -| f) $){};
			\node[gate](igate) at ($ (gates -| i) $){};
			\node[gate](jgate) at ($ (gates -| j) $){};
			\node[gate](ogate) at ($ (gates -| o) $){};
			
%			\coordinate (htminus1) at bl+(-.5,0);
%			\coordinate (ht) at bl+(-.5,0);
%			
			% forget gate
			\node[operator](fmult) at ($ (bl -| fgate) $) {};
			\draw[endflow] (-input) -| (fgate) -- (fmult); 
			
%			%j
			\node[operator](jmult) at ([shift={(0,-1*\h)}]jgate) {};
			\node[operator](cadd) at ($ (bl -| jgate) $) {};
			\draw[endflow] (-input) -| (jgate) -- (jmult);
			\draw[endflow] (jmult) -- (cadd); 			

%			%i	
			\draw[endflow] (-input) -| (igate) |- (jmult); 
%
%%			% outpu
			\node[operator](outtanh) at ([shift={(0,1*\h)}]out) {};
%			
%			%o 
			\draw[endflow] (tl) -| (ogate) |- (outtanh);
			\draw[flow] (outtanh) |- (-houtput);
%			
%			% output flow
			\draw[endflow] (cadd) -| (outtanh);
			\draw[flow] (-cinput) -- (fmult) -- (cadd) -- (-coutput);
%			
			
			% debug
%			\node at (gates) {\tiny{gates}};
%			\node at (-input) {\tiny{-input}};
%			\node at (-coutput) {\tiny{-coutput}};
%			\node at (-houtput) {\tiny{-houtput}};
%			\node at (f) {\tiny{f}};
%			\node at (i) {\tiny{i}};
%			\node at (j) {\tiny{j}};
%			\node at (o) {\tiny{o}};
%			\node at (tl) {\tiny{tl}};
%			\node at (br) {\tiny{br}};
%			\node at (bl) {\tiny{bl}};
%			\node at (tr) {\tiny{tr}};
%			\node at (out) {\tiny{out}};
			
		\end{scope}
		}
	}
}

\tikzset{pic shift/.store in=\shiftcoord,
	pic shift={(0,0)},
	pics/seqlstmbw/.style={
		code={
			\begin{scope}[shift={\shiftcoord},xscale=1.3,yscale=-.8]
				
				\node[dummy] (bl) at (0,0){}; % bottom left
				\node[dummy] (tr) at (1,1){}; % top right
				
				\node[dummy] (br) at ($ (bl -| tr) $){}; % bottom right
				\node[dummy] (tl) at ($ (bl |- tr) $){}; % top left
				
				\node[fit=(bl) (tr),bigbox] (-C) {};
				
				% input coordinate for rounded draw lines -> slightly right of tl
				\coordinate (-input) at (0.1,1); % top left
				
				% output coordinate for rounded draw lines -> slightly left of br
				\coordinate (-coutput) at (0.9,0); % bottom right
				\coordinate (-cinput) at (0.1,0); % bottom left
				\coordinate (-houtput) at (0.9,1); % top right
				
				%			% gate distance
				\def\d{1/6}
				
				% gate heights
				\def\h{1/3}
				
				\coordinate (f)  at bl+(1*\d,0);
				\coordinate (i)  at bl+(2*\d,0);
				\coordinate (j)  at bl+(3*\d,0);
				\coordinate (o)  at bl+(4*\d,0);
				\coordinate (out) at bl+(5*\d,0);
				
				\coordinate (gates) at (0,2*\h);
				
				%\node[above=of tl](xt){$x_{t}$};
				%\node[left=of tl](htminus1){$h_{t-1}$};
				
				%\node[below=of br](ct){$c_{t}$};
				
				\node[gate](fgate) at ($ (gates -| f) $){};
				\node[gate](igate) at ($ (gates -| i) $){};
				\node[gate](jgate) at ($ (gates -| j) $){};
				\node[gate](ogate) at ($ (gates -| o) $){};
				
				%			\coordinate (htminus1) at bl+(-.5,0);
				%			\coordinate (ht) at bl+(-.5,0);
				%			
				% forget gate
				\node[operator](fmult) at ($ (bl -| fgate) $) {};
				\draw[endflow] (-input) -| (fgate) -- (fmult); 
				
				%			%j
				\node[operator](jmult) at ([shift={(0,-1*\h)}]jgate) {};
				\node[operator](cadd) at ($ (bl -| jgate) $) {};
				\draw[endflow] (-input) -| (jgate) -- (jmult);
				\draw[endflow] (jmult) -- (cadd); 			
				
				%			%i	
				\draw[endflow] (-input) -| (igate) |- (jmult); 
				%
				%%			% outpu
				\node[operator](outtanh) at ([shift={(0,1*\h)}]out) {};
				%			
				%			%o 
				\draw[endflow] (tl) -| (ogate) |- (outtanh);
				\draw[flow] (outtanh) |- (-houtput);
				%			
				%			% output flow
				\draw[endflow] (cadd) -| (outtanh);
				\draw[flow] (-cinput) -- (fmult) -- (cadd) -- (-coutput);
				%			
				
				% debug
				%			\node at (gates) {\tiny{gates}};
				%			\node at (-input) {\tiny{-input}};
				%			\node at (-coutput) {\tiny{-coutput}};
				%			\node at (-houtput) {\tiny{-houtput}};
				%			\node at (f) {\tiny{f}};
				%			\node at (i) {\tiny{i}};
				%			\node at (j) {\tiny{j}};
				%			\node at (o) {\tiny{o}};
				%			\node at (tl) {\tiny{tl}};
				%			\node at (br) {\tiny{br}};
				%			\node at (bl) {\tiny{bl}};
				%			\node at (tr) {\tiny{tr}};
				%			\node at (out) {\tiny{out}};
				
			\end{scope}
		}
	}
}

%\newcommand{}{
%	\begin{tikzpicture}
%	% each layer
%	\begin{scope}[scale=2]
%	
%	% raster size
%	\def\d{0.7}		
%	
%	% distance layer
%	\def\s{\d*5}
%	
%	\foreach \i in {1,...,6}
%	{		
%		\begin{scope}[
%		yshift=\s*\i,every node/.append style={
%			yslant=0.5,xslant=-1},yslant=0.5,xslant=-1
%		]
%		%\draw[step=3.33mm] (0,0) grid (1,1);
%		%\fill[black,fill opacity=.9] (0.333,0.333) rectangle (0.333,0.333);    	    	  
%		
%		\foreach \row in {0,...,2}{
%			\foreach \col in {0,...,2}{
%				\draw[tumblack, fill=tumblue!\pdfuniformdeviate 40,fill opacity=1,rounded corners=1] (\col*\d/3,\row*\d/3) rectangle (\col*\d/3+\d/3, \row*\d/3+\d/3);
%				%                 \draw[black, fill=black!\pdfuniformdeviate 40,fill opacity=1,rounded corners=1] (\col*\d/3,\row*\d/3) rectangle (\col*\d/3+\d/3, \row*\d/3+\d/3);
%			}
%		}
%		
%		%\draw[step=3.33mm] (0,0) grid (1,1);
%		%\fill[white,fill opacity=.9] (0,0) rectangle (1,1);
%		\end{scope}
%	}
%	\end{scope}
%	\end{tikzpicture}
%}
%%
%\newcommand{}{
%	
%	\begin{tikzpicture}[scale=2]
%	\foreach \i in {0,...,9}
%	\draw[tumblack,fill=tumorange!\pdfuniformdeviate 40,fill opacity=.9, rounded corners=0.5] (\i*.2, 0) rectangle (\i*.2+.2, .2);
%	\end{tikzpicture}
%}

\newcommand{\figencoderfieldRNN}{
	
	\begin{tikzpicture}[scale=1, node distance=1em]%,show background rectangle,background rectangle/.style={draw=red}]
	
	%\matrix (m) [matrix of nodes, ampersand replacement=\&]{
	
	\def\d{1.7}
	\def\encoderheight{0.4}
	\def\decoderheight{-0.4}
	\def\imagewidth{8mm}
	\def\stateimagewidth{8mm}
	\def\classimagewidth{8mm}
	
	\def\rgbone{images/network/48px/rgb1}
	\def\rgbtwo{images/network/48px/rgb2}
	\def\rgbthree{images/network/48px/rgb3}
	\def\rgbfour{images/network/48px/rgb4}
	\def\prediction{images/network/48px/prediction}
	\def\groundtruth{images/network/48px/ground_truth}
	\def\activation{images/network/48px/maize}
	\def\state{images/network/48px/state}
	
	
	%\draw [bigpassbox, fill=forwardcolor](1,.25) rectangle (10,4);
	
	\node[bigpassbox, fill=none, rectangle, minimum width=7cm,minimum height=3cm, anchor=south west, label={[shift={(2em,-3.5ex)}]north:}] (forwardbox) at (1,.15) {};
	%\draw [bigpassbox, fill=backwardcolor](1,-.25) rectangle (10,-4);
	
	\draw pic (fw1) at (\d,\encoderheight) {seqlstmfw};% \&
	\node[above=of fw1tl, label=above:{$\VInput_{1}$}](xfw1){
		
	};%images/network/time1_x
	
	\node[below= 3.5emof fw1-houtput, xshift=.5em, label=below:{$\V{y}_{1}$}](yfw1){
		
	};%images/network/time1_x
	
	\draw pic (fw2) at (2*\d,\encoderheight) {seqlstmfw};
	\node[above=of fw2-input, label=above:{$\VInput_{2}$}](xfw2){
		
	}; % images/network/time11_x
	
	\node[below= 3.5emof fw2-houtput, xshift=.5em, label=below:{$\V{y}_{2}$}](yfw2){
		
	};%images/network/time1_x
	
	\draw pic (fw3) at (3*\d,\encoderheight) {seqlstmfw};
	\node[above=of fw3-input](xfw3){$\dots$};
	\node[below= 3.5emof fw3-houtput, xshift=.5em](yfw3){
		%	
		$\dots$
	};%images/network/time1_x
	
	\draw pic (fw4) at (4*\d,\encoderheight) {seqlstmfw};
	\node[above=of fw4-input, label=above:{$\VInput_T$}](xfw4){
		
	}; % images/network/time30_x
	
	\node[below= 3.5emof fw4-houtput, xshift=.5em, label=below:{$\V{y}_{T}$}](yfw4){
		
	};%images/network/time1_x
	
	\node[left=of fw1-input](enczerostateh){$\V{0}$};
	\node[left=of fw1-cinput](enczerostatec){$\V{0}$};
	\draw[endflow] (enczerostateh) -- (fw1-input);
	\draw[endflow] (enczerostatec) -- (fw1-cinput);
	
	% draw connections from input to cells
	\draw[flow] (xfw1) |- (fw1-input);
	\draw[flow] (xfw2) |- (fw2-input);
	\draw[flow] (xfw3) |- (fw3-input);
	\draw[flow] (xfw4) |- (fw4-input);
	
	% draw hidden connections between cells
	\draw[endflow] (fw1-houtput) -- (fw2-input);
	\draw[endflow] (fw2-houtput) -- (fw3-input);
	\draw[endflow] (fw3-houtput) -- (fw4-input);
	
	% draw hidden connections between cells
	\draw[endflow] (fw1-coutput) -- (fw2-cinput);
	\draw[endflow] (fw2-coutput) -- (fw3-cinput);
	\draw[endflow] (fw3-coutput) -- (fw4-cinput);
	
	% outputs
	\draw[endflow] (fw1-houtput) -- ($ (fw1-houtput)+(.5em,0) $) -- (yfw1);
	\draw[endflow] (fw2-houtput) -- ($ (fw2-houtput)+(.5em,0) $) -- (yfw2);
	\draw[endflow] (fw3-houtput) -- ($ (fw3-houtput)+(.5em,0) $) -- (yfw3);
	\draw[endflow] (fw4-houtput) -- ($ (fw4-houtput)+(.5em,0) $) -- (yfw4);
	
	%\node[right= 2.5em of fw4-coutput](cfw){$\VCellState_T^\text{fw}$};
	
	%\draw[endflow] (fw4-coutput) -- (cfw);
	
	%};
	\end{tikzpicture}
}

\newcommand{\figencoder}{

\begin{tikzpicture}[scale=1, node distance=1em]%,show background rectangle,background rectangle/.style={draw=red}]

%\matrix (m) [matrix of nodes, ampersand replacement=\&]{

\def\d{1.7}
\def\encoderheight{0.4}
\def\decoderheight{-0.4}
\def\imagewidth{8mm}
\def\stateimagewidth{8mm}
\def\classimagewidth{8mm}

\def\rgbone{images/network/48px/rgb1}
\def\rgbtwo{images/network/48px/rgb2}
\def\rgbthree{images/network/48px/rgb3}
\def\rgbfour{images/network/48px/rgb4}
\def\prediction{images/network/48px/prediction}
\def\groundtruth{images/network/48px/ground_truth}
\def\activation{images/network/48px/maize}
\def\state{images/network/48px/state}


%\draw [bigpassbox, fill=forwardcolor](1,.25) rectangle (10,4);

\node[bigpassbox, fill=none, rectangle, minimum width=7cm,minimum height=3cm, anchor=south west, label={[shift={(2em,-3.5ex)}]north:}] (forwardbox) at (1,.15) {};
%\draw [bigpassbox, fill=backwardcolor](1,-.25) rectangle (10,-4);

\draw pic (fw1) at (\d,\encoderheight) {seqlstmfw};% \&
\node[above=of fw1tl, label=above:{$\VInput_{1}$}](xfw1){\includegraphics[width=\imagewidth]{\rgbone}};%images/network/time1_x

\draw pic (fw2) at (2*\d,\encoderheight) {seqlstmfw};
\node[above=of fw2-input, label=above:{$\VInput_{2}$}](xfw2){\includegraphics[width=\imagewidth]{\rgbtwo}}; % images/network/time11_x

\draw pic (fw3) at (3*\d,\encoderheight) {seqlstmfw};
\node[above=of fw3-input](xfw3){$\dots$};

\draw pic (fw4) at (4*\d,\encoderheight) {seqlstmfw};
\node[above=of fw4-input, label=above:{$\VInput_T$}](xfw4){\includegraphics[width=\imagewidth]{\rgbfour}}; % images/network/time30_x

\node[left=of fw1-input](enczerostateh){$\V{0}$};
\node[left=of fw1-cinput](enczerostatec){$\V{0}$};
\draw[endflow] (enczerostateh) -- (fw1-input);
\draw[endflow] (enczerostatec) -- (fw1-cinput);

% draw connections from input to cells
\draw[flow] (xfw1) |- (fw1-input);
\draw[flow] (xfw2) |- (fw2-input);
\draw[flow] (xfw3) |- (fw3-input);
\draw[flow] (xfw4) |- (fw4-input);

% draw hidden connections between cells
\draw[endflow] (fw1-houtput) -- (fw2-input);
\draw[endflow] (fw2-houtput) -- (fw3-input);
\draw[endflow] (fw3-houtput) -- (fw4-input);

% draw hidden connections between cells
\draw[endflow] (fw1-coutput) -- (fw2-cinput);
\draw[endflow] (fw2-coutput) -- (fw3-cinput);
\draw[endflow] (fw3-coutput) -- (fw4-cinput);

\node[right= 2.5em of fw4-coutput](cfw){$\VCellState_T^\text{fw}$};

\draw[endflow] (fw4-coutput) -- (cfw);

%};
\end{tikzpicture}

}


\begin{frame}
\frametitle{Classification ConvLSTM Network}
\input{images/network.tikz}
\end{frame}


\begin{frame}

\vfill
\Huge\color{black}
\begin{center}
	\begin{columns}
		\column{\textwidth}
		\vspace{7em}
		
		\textbf{Surprise:} 
		\hfill It worked without specifically labeling clouds!
%		\only<1>{Surprise:} \only<2>{It worked without labeling clouds!}
		%			\includegraphics[width=5cm]{images/dscovrepic/epic1}
		%\includegraphics[width=7cm]{images/fdl}
	\end{columns}
\end{center}

\vfill

\end{frame}


\begin{frame}
\frametitle{ConvLSTM robust to clouds}
\input{images/scl.tikz}
\input{images/clouds.tikz}
\end{frame}


\begin{frame}
	\frametitle{Remembering Karpathy's "Unreasonable Effectiveness of RNNs"}
	
	\includegraphics[width=\textwidth]{images/karpathy}
	
	\url{http://karpathy.github.io/2015/05/21/rnn-effectiveness/}
\end{frame}


\input{images/activations.tikz}
\begin{frame}
\frametitle{Internal States Encode increasingly Classification Features}
\framesubtitle{LSTM cell \textbf{47} of 256}
\figactivations{1}{3}
\end{frame}
%

\begin{frame}
\frametitle{Found Cloud Masking Cells in the RNN}
\framesubtitle{LSTM cell \textbf{47} of 256}
\figactivations{1}{47}
\end{frame}
%
%\begin{frame}
%	\frametitle{Publication and Github}
%	
%			Including the spatial information by adding convolutions
%	\vspace{2em}
%	{\small
%		Rußwurm, M., \& Körner, M. (2018). Multi-temporal land cover classification with sequential recurrent encoders. ISPRS International Journal of Geo-Information, 7(4), 129.
%	}
%
%	Github
%	\url{https://github.com/tum-lmf/mtlcc}
%	
%\end{frame}


\begin{frame}[c]
\frametitle{Paper and Code}
\centering 

%	\vspace{3em}

\large



Github + DockerHub + Continuation with GAF AG

\vspace{1ex}

\includegraphics[width=2cm]{images/github} \hspace{.5ex}
\includegraphics[width=2cm]{images/qr_github} \hspace{.5ex}
\includegraphics[width=2cm]{images/docker} \hspace{.5ex}
\vline
\hspace{.5ex}
\includegraphics[width=4cm]{images/gaf}

\vspace{1ex}

\url{https://github.com/TUM-LMF/MTLCC}
\url{https://github.com/TUM-LMF/MTLCC-pytorch}

\url{http://www.lmf.bgu.tum.de/vision/}

\vspace{1em}
\begin{columns}[t]
	\column{.5\textwidth}
	\scriptsize
	\textsl{
		Rußwurm, M. and Körner, M. (2017). \textbf{Temporal Vegetation Modelling using Long Short-Term Memory Networks for Crop Identification from Medium-Resolution Multi-Spectral Satellite Images}. In IEEE/ISPRS EarthVision 2017 Workshop, Proceedings of the IEEE CVPR Workshops.
	}
	
	\column{.5\textwidth}
	\small
	\textsl{
		Rußwurm M., Körner M. (2018). \textbf{Multi-Temporal Land Cover Classification with Sequential Recurrent Encoders}. ISPRS International Journal of Geo-Information. https://arxiv.org/abs/1802.02080. (in review)
	}
	
\end{columns}


\end{frame}

%\begin{frame}
%\frametitle{Found Cloud Masking Cells in the RNN}
%\framesubtitle{LSTM cell \textbf{47} of 256}
%\figactivations{1}{3}
%\end{frame}
