%
%\begin{frame}
%\frametitle{Between two fields}
%\centering
%\begin{tikzpicture}[scale=2]
%%\draw[fill=tumblue, draw=none, opacity=0.5](-1,0) circle (1.5);
%\node[fill=tumbluelight, draw=none, opacity=0.5, circle, minimum width=6cm, label=Earth Observation] at (-1,0){};
%
%\node[fill=tumorange!50, draw=none, opacity=0.5, circle, minimum width=6cm, label=Machine Learning] at (1,0){};
%%\draw[fill=tumorange, draw=none, opacity=0.5](1,0) circle (1.5);
%
%\draw[-stealth, shorten >=1cm] (-1.3,1) -- (0,0);
%\draw[-stealth, shorten >=1cm] (-1.6,-1) -- (0,0);
%\draw[-stealth, shorten >=1cm] (-1.2,-.6) -- (0,0);
%\draw[-stealth, shorten >=1cm] (-1.8,.4) -- (0,0);
%\draw[-stealth, shorten >=1cm] (-1.3,.2) -- (0,0);
%
%\node[font=\bfseries] at (-2,0) {applications};
%
%\draw[-stealth, shorten >=1cm] (1.3,1) -- (0,0);
%\draw[-stealth, shorten >=1cm] (1.6,-1) -- (0,0);
%\draw[-stealth, shorten >=1cm] (1.2,-.6) -- (0,0);
%\draw[-stealth, shorten >=1cm] (1.8,.4) -- (0,0);
%\draw[-stealth, shorten >=1cm] (1.3,.2) -- (0,0);
%
%\node[font=\bfseries] at (2,0) {methods};
%
%\node[text width=2cm, circle, fill=tumblue, text=white]{global scale};
%
%\node[text width=2cm, circle, fill=tumblue, text=white]{global scale};
%
%\node[font=\normalsize, fill=white, text width=2cm, rounded corners, fill opacity=.5, text opacity=1](phd) at (0,0){global scaleability \\ real world impact \\ Open Data};
%
%%\draw[-stealth, very thick] (phd) -- (0,, fil0);
%\end{tikzpicture}
%
%
%\end{frame}


\begin{frame}
\frametitle{Multi-temporal Earth observation}
\centering
\begin{tikzpicture}[scale=2]
%\draw[fill=tumblue, draw=none, opacity=0.5](-1,0) circle (1.5);
\node[fill=tumgraylight, draw=none, opacity=0.5, circle, minimum width=6cm, label=Earth Observation] at (-1,0){};

\node[fill=tumgraylight, draw=none, opacity=0.5, circle, minimum width=6cm, label=Machine Learning] at (1,0){};

\visible<1->{
\node[font=\bfseries, circle, fill=tumbluelight, text width=2.5cm] (vhr) at (-1.5,.7) {high spatial \\ resolution};
\node[font=\bfseries, circle, fill=tumorange!50, text width=2.5cm] (cv) at (1.5,.7) {computer vision methods};
\draw[stealth-stealth, very thick] (vhr) -- node[midway,above]{well established} (cv);
}

\visible<2->{
\node[font=\bfseries, circle, fill=tumbluelight, text width=2.5cm] (mt) at (-1.5,-.7) {high temporal resolution};
\node[font=\bfseries, circle, fill=tumorange!50, text width=2.5cm] (nlp) at (1.5,-.7) {natural \\ language \\ processing};
\draw[stealth-stealth, dotted] (mt) -- node[midway,above]{hardly anyone} (nlp);
}

\visible<3->{
\node[fit=(nlp)(mt), draw, inner sep=.5em, rounded corners, thick, label=above:{\bfseries \Large my focus}]{};
}
%\draw[-stealth] (cv) -- (0,0);

%\draw[-stealth, very thick] (phd) -- (0,, fil0);
\end{tikzpicture}

\end{frame}


\begin{frame}
\frametitle{Example Analogy to Natural Language Processing}
\setrand{0}{100}{0.01}{1}
	\newcommand{\drawmatrix}{
		\left(\begin{matrix}\nextrand\thisrand\\\nextrand\thisrand\\\nextrand\thisrand\end{matrix}\right)
	}

	\newcommand{\image}[1]{
		\begin{tikzpicture}
			\node(img){\includegraphics[width=2cm]{#1}};
			\node[minimum width=.5em,minimum height=.5em, fill=tumbluelight, xshift=-1em] at (img)(rect){};
			
			\node[right=of rect, fill=tumbluelight,, inner sep=.2em, rounded corners=1em,  opacity=.2](m){$\drawmatrix$};
			\draw[tumbluelight] (rect.north) -- (m.north);
			\draw[tumbluelight] (rect.south) -- (m.south);
		\end{tikzpicture}
	}

	
	\begin{tikzpicture}[node distance=1em]
		\node[font=\scriptsize](e1){\image{images/analogy_examples/170127_snow.png}};
		\node[right=of e1, font=\scriptsize](e2){\image{images/analogy_examples/160929_clear.png}};
		\node[right=of e2, font=\scriptsize](e3){\image{images/analogy_examples/161115_cloudy.png}};
		\node[right=of e3, font=\scriptsize](e4){\image{images/analogy_examples/160728_partlycloudy.png}};
		
		\node[below=of e1, font=\scriptsize](t1){$E(\text{\textbf{The}})=\drawmatrix$};
		\node[below=of e2, font=\scriptsize](t2){$E(\text{\textbf{eagle}})=\drawmatrix$};
		\node[below=of e3, font=\scriptsize](t3){$E(\text{\textbf{has}})=\drawmatrix$};
		\node[below=of e4, font=\scriptsize](t4){$E(\text{\textbf{landed}})=\drawmatrix$};
		
		\node[above=of e1](x1){$\V{x}_1$};
		\node[above=of e2](x2){$\V{x}_2$};
		\node[above=of e3](x3){$\V{x}_3$};
		\node[above=of e4](x4){$\V{x}_4$};
		
		\node[right= 7em of e4](eo){$f(\M{X})$};
		\node[right= 7em of t4](nlp){$f(\M{X})$};
		
		\draw[-stealth] (e4) -- node[midway,above]{EO model} (eo);
		\draw[-stealth] (t4) -- node[midway,above]{NLP model} (nlp);
	\end{tikzpicture}
\end{frame}
%



%
%
%\begin{frame}
%	\frametitle{Natural Language Processing}
%	
%	GPT-2 
%%	\cite{radford2019language}
%	
%	Bert Model Pretraining
%%	\cite{Devlin2018bert}
%	
%	
%\end{frame}

